\section*{Problem 1: Mixed Integer Programming (MIP)}

\newcommand{\SolverMIP}{\todo{Gurobi}\xspace}  %% or CPLEX, or Xpress, or Cbc
\newcommand{\TimeoutMIP}{\todo{314.56}}  %% timeout, in CPU seconds; MIN 300.00

\paragraph{Task~A.}~
\todo{What are the variables, their meanings, their constraints, and
  the objective function?
  % 
  For example, for the investment design problem, one might write:
  ``Let variable~$m_{ij}$ take value~$1$ if basket~$i$ invests in
  credit~$j$, and value~$0$ otherwise, with $i \in 1 \twodots v$ and
  $j \in 1 \twodots b$.
  % 
  The constraint that each row must sum up to~$r$ can then be linearly
  modelled as
  \[
    \forall i \in 1 \twodots v : \sum_{j \in 1 \twodots b} m_{ij} = r
  \]
  % 
  Etc.''
  %
}

(We are aware that we may lose~1 point if the model described above is
different from what is actually implemented in Task~B and evaluated in
Tasks~C to~H.)

\paragraph{Task~B.}
Our model \texttt{servStatLoc.mod} is uploaded with this report (but
not listed inside it): we checked that its constraints and objective
function are linear (and we are aware that~$4$~points will otherwise
be deducted from our score for this problem).

We chose the MIP solver~\SolverMIP for our experiments, which we ran
\todo{on the NEOS server or
  %% Below is a specification of the ThinLinc Linux hosts of the IT dept:
  %% --> Replace by a similar-looking specification of your used hardware
  %% (under Linux, do lscpu to find the specification, and under
  %% macOS, you find it via "About This Mac" in the Apple menu):
  under Linux Ubuntu~22.04.5 ($64$~bit) on an Intel Xeon E5520 of
  $2.27$~GHz, with $4$~processors of $4$~cores each, with a $70$~GiB
  RAM and an $8$~MiB L3 cache (a ThinLinc computer of the IT
  department).}  % barany.it.uu.se

\paragraph{Tasks C, D, E, and F.}
The results are in Table~\ref{tab:res:mip}.

\paragraph{Task G.}
The results are in Table~\ref{tab:res:mip}.
%
Our model \todo{does not time out}.

\paragraph{Task H.}
The results are in Table~\ref{tab:res:mip}.
%
Our model \todo{times out,
%% Optional for solo teams:
  so our proposed algorithm for delivering a not necessarily optimal
  solution in reasonable running time is \dots\ as follows \dots}.

%% Optional for solo teams:
\paragraph{Task~I.}
The size of the search space of the problem is
\todo{$\binom{z!}{\cos c} \cdot \log_s v$}, because \todo{\dots\
  justification \dots}.

The numbers of candidate solutions this brute-force search algorithm
has to examine per second in order to match the reported runtime
performance of~\SolverMIP on our model are given in the right-most
column of Table~\ref{tab:res:mip}, for each instance that~\SolverMIP
solved to proven optimality without timing out.

\begin{table}[t]
  \centering
  \begin{tabular}{rrrrrrrr}
    $z$ & $s$ & $v$ & $c$ & time & objective value & optimality gap & brute-force \\
    \midrule
    %% Make sure every number in a column has the _same_ number of decimals,
%% so as to get decimal-point alignment and easy comparison of numbers!
%%
%% Witness in particular the 0.023246261350 instead of 0.02324626135.
%%
 10 &  2 & 2 & 3 & 0.01 & 0.008738745537 & 0.00\% & n/a \\
 10 &  3 & 2 & 3 & 0.01 & 0.007145209158 & 0.00\% & n/a \\
 10 &  4 & 2 & 3 & 0.01 & 0.005884012648 & 0.00\% & n/a \\
 20 &  2 & 2 & 3 & 0.02 & 0.02324309492 & 0.00\% & n/a \\
 20 &  3 & 2 & 3 & 0.03 & 0.01668708501 & 0.00\% & n/a \\
 20 &  4 & 2 & 3 & 0.04 & 0.01391381432 & 0.00\% & n/a \\
 20 &  5 & 2 & 3 & 0.02 & 0.01191855424 & 0.00\% & n/a \\
 20 &  6 & 2 & 3 & 0.04 & 0.010842829501 & 0.00\% & n/a \\
 40 &  5 & 2 & 3 & 0.31 & 0.048476076118 & 0.00\% & n/a \\
 80 &  8 & 2 & 3 & 3.07 & 0.124547186911 & 0.00\% & n/a \\
 80 & 16 & 1 & 3 & 0.10 & 0.116981746321 & 0.00\% & n/a \\
120 & 10 & 2 & 3 & 9.94 & 0.232382635834 & 0.01\% & n/a \\
250 & 12 & 3 & 4 & \TimeoutMIP & -- & -- & n/a \\ %% Let your experiment script write directly
                            %% into this file, making sure every number
                            %% in a column has the _same_ number of decimals
  \end{tabular}
  \caption{Service station location: runtime (in seconds), objective
    value, and optimality gap (in percent; positive if an optimal
    solution was not found and proven before timing out)
    using~\SolverMIP, with a timeout of $\TimeoutMIP$~CPU seconds.
    The right-most column gives the number of candidate solutions a
    totally brute-force search algorithm has to examine per second in
    order to match the runtime performance of~\SolverMIP, if the
    instance was solved to proven optimality, and~`n/a' for
    `non-applicable' otherwise.
    %% Delete the following sentence:
    \todo{(The sample performance of this skeleton table is made up,
      but the two given optimal objective values are correct!)}
    % 
  }
  \label{tab:res:mip}
\end{table}

%%% Local Variables:
%%% mode: latex
%%% TeX-master: "assignment1-report"
%%% End:
